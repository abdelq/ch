\documentclass[a4paper]{article}

\usepackage[utf8]{inputenc} % pour les accents UTF-8
\usepackage{ragged2e}       % justify paragraphs
\usepackage[margin=0.9in]{geometry}
\usepackage{listings}

\title{Travail pratique \#1 -- IFT-2245}
\author{Abdelhakim Qbaich & Rémi Langevin}
\date{\today}

\begin{document}
\maketitle

\section*{Probl\`emes rencontr\'es}
    \subsection*{G\'erer les assignements de variables multiples}
        \'Etant donn\'e que nous voulions \'eviter de g\'erer les erreurs d\^ues \`a des caract\`eres invalides
        lors de l'utilisation des variables, nous avons utiliser des expressions r\'eguli\`eres afin de 
        "whitelist" les caract\`eres valides. Toutefois, cela a rendu plus hardu la gestion de variables de `shell`.
        Ainsi la commande suivante n'est pas valide:
        \begin{lstlisting}[language=bash]
PATH=miaw HOME=wouf echo $HOME $PATH
        \end{lstlisting}
        Toutefois, il est possible d'assigner multiples variables d'environnement comme suit :
        \begin{lstlisting}[language=bash]
COCO=lapin MIAW=RonRonRon
        \end{lstlisting}

    \subsection*{Boucle "for"}
        Plusieurs id\'ees nous sont venues en t\^ete afin d'impl\'ementer la boucle "for". Nous avons d'abord
        essayer d'avoir des triples pointeurs. Toutefois, cela s'av\`ere affreux et complexe \`a g\'erer.
        Nous avons ensuite essayer d'avoir un struct contenant des pointeurs vers le d\'ebut de chaque partie
        d'une boucle "for" (i.e. variable d'it\'eration, valeurs et le corps de la boucle). Cela s'est av\'er\'e
        bien mieux et nous avons peaufiner le tout autour de cette m\'ethode. Ensuite, avec quelque peu de
        r\'ecursion, nous avons r\'eussi \`a g\'erer les for imbriqu\'es.
    \subsection*{"cd -"}
        Pour le moment, nous ne supportons pas la commande "cd -", pour revenir au r\'epertoire pr\'ec\'edent
        (temporellement). Toutefois, il ne s'agit que de bien g\'erer et mettre \`a jour les variables PWD et 
        OLDPWD.
    %\subsection*{Fuites de m\'emoire dans "expand"}
    %Initialement arg pointe vers de la mémoire dans le "stack". Puis par la suite,
    %arg pointe sur le "heap", dû à l'expansion au \textit{runtime} des variables
    %d'environnement que l'on concatène à d'autres chaînes de caractères.
    %Le seul problème, est que l'on ne peut pas nécessairement appeler free

\section*{Surprises}
    Le travail a été plus ardu qu'initialement prévu.
    Le manque de temps n'a pas aidé.
    La manipulation de string n'est pas si aisée que ça, comparativement à des
    langages plus haut niveau.
    %\subsection*{Devoir utiliser C}
    %Nous aurions voulu utiliser C++ ou encore mieux: {\bf Rust}. Safety first!

\section*{Choix}
    \subsection*{Variables d'environnement non-existante}
    Lorsque que le shell a une requ\^ete pour retrouver une variable
    d'environnement qui n'existe pas, nous retournons une string vide plut\^ot
    que de l'ignorer.
        \begin{lstlisting}[language=bash]
echo $HOME $MIAW $HOME
        \end{lstlisting}

    \subsection*{Gestion du CTRL+C}
        Nous avons choisi de ne pas catch les SIGINT g\'en\'er\'es par les touches CTRL+C. Ainsi, le shell est 
        interrompu plut\^ot que le "child process" seulement.

\section*{Options rejet\'ees}
    \subsection*{Yacc/Lex Bison/Flex}
    L'idée originale avec ce projet était d'utiliser Yacc/Lex ou Bison/Flex pour
    générer une grammaire et son parseur, ce qui simplifierait énormément la tâche,
    considérant que le parsing des expressions et un des éléments les plus compliqués,
    surtout lorsqu'on parle de récursitivité dans les boucles.

    Cependant, par manque de temps, l'étude de la syntaxe byzantine de Yacc/Lex
    n'était pas une si bonne option, surtout pour un shell relativement de base.

    En conséquent, nous avons utiliser strsep(3). Un choix plutôt typique, mais
    qui représente une légère amélioration par rapport à l'utilisation de 
    strtok(3).

    \subsection*{Commandes built-in \`a l'int\'erieur d'une boucle "for"}
    Nous avons rejet\'e l'option de pouvoir ex\'ecuter des commandes tel que "cd" dans une boucle "for".
    \break

    Outre cela, nous n'avons pas particuli\`erement rejet\'e d'options, \`a
    l'exception d'options avanc\'ees (e.g. completion/suggestion, pipe, etc.)

\section*{Ajouts}
    \subsection*{Bindings emacs et historique}
        Avec la libraire readline, nous avons pu facilement avoir les bindings
        emacs et g\'erer l'historique de la session. De plus, cela permet de
        d\'eplacer le curseur sur la ligne de commande.
    \subsection*{Easter Egg}
        Essayer les commandes suivantes:
        \begin{lstlisting}[language=bash]
twado                       # pour souligner le twadoversary
sudo rm -rf --no-preserve-root /   # pour du bon temps
        \end{lstlisting}

\end{document}
